\documentclass{article}
\usepackage[utf8]{inputenc}

\title{FOAR705 Elaboration}
\author{Bart Wojcik}
\date{6 September 2019}

\begin{document}

\maketitle

\section{Introduction}
Upon consultation with Brian (I thought my initial project was trivial to accomplish), I have decided to expand the scope my project and the final goal will be to pre-configure and pre-package a ready for distribution and deployment collaborative work environment for researchers and students working with digital texts in Japanese language. The initial goal to make it easier and quicker to complete multiple dictionary look-ups will still remain a large part of the project. The expanded functionality will include annotation and collaboration functionality that would allow, for example, a research student and supervisor to annotate, markup and collaboratively communicate while working on a text from using a pre-configured digital work space.

\section{Project Requirements}
\begin{enumerate}
    \item Self-contained digital work space
    \item Built-in enhanced dictionary
    \item Annotation or mark-up functionality
    \item Share-ability of annotations between collaborators
\end{enumerate}
\section{Packaging and Deployment Requirements and Options}
\begin{enumerate}
    \item Final research environment must be prepacked to reduce set-up by user to minimum
    \item Final package must offer the option for portable (USB) and local (local installation) deployments
    \item Final environment must be multi-platform (at the moment I am looking at packaging for Windows and Linux systems, in the case of Linux I am limiting myself to Debian and derivative distributions such as Ubuntu, Mint etc. I will look into the possibility of prepackaging for Mac OS.)
\end{enumerate}

\section{Technology Required}
\begin{enumerate}
    \item Virtual Machine - I need to be able to develop and test on Windows, Linux and MacOSX (or whatever they call it these days, I'll stick to calling it MacOSX for now). Oracle VirtualBox should suffice, I am already using it to run Ubuntu 19.04 on a Windows 10 host which takes care of two of my target operating systems. I just need to get my hands on a MacOSX iso which looks to be obtainable upon some initial web searches.
    \item Target Platforms - Windows, Linux, MacOS
    \item Firefox browser - Firefox is a flexible and open source web browser which offers binary executables for my target platforms. Firefox will form the basis and the core user interface of the final environment, I will need a way to remove and replace the Firefox branding and replace it with something specific to this project, for the sake of non-technical user sanity, I don't want the end user to be deploying a Firefox branded package. If the user already has Firefox installed, this could cause confusion. I will be looking into preparing a package in a way similar to the way Tor Project (https://www.torproject.org/) packages theirs. I will look at packaging this up in such a way that when a portable installation is used, it can be executed on all three of my target platforms without any loss of configuration or work space data.
    \item Japanese - English hover dictionary plugins for Firefox - I would like the final environment to allow the user to choose between a faster and more stable Rikaichamp dictionary and the more customisable Yomichan dictionary plugin for power users.
    \item Annotation solution - I have two candidates here: Hypothesis \newline (https://web.hypothes.is/) and Liner (https://getliner.com/)
    \item Packaging and deployment solution for the final package. I believe that Mozilla Institutional Deployment \newline (https://wiki.mozilla.org/Firefox:2.0\verb|_|Institutional\verb|_|Deployment) should supply some initial ideas if not a solution.
    \item Any other collaboration tools which I may come across in the coming month that can be integrated into the environment.
\end{enumerate}

\section{Goals}
In simple terms, the goal is to give a non-technical user working with digital text in Japanese a pre-configured software package that makes it easier to read the text and collaborate on it, requires minimal technical knowledge to set-up and works on common operating systems out of the box.

\section {Technology Testing}
\subsection{Virtual Machine}
\begin{enumerate}
    \item QEMU - I have prior experience with QEMU. Almost a decade ago, I was using QEMU to emulate x86 architecture processors on a PowerPC system. I have found QEMU to be flexible but not as robustly supported as my chosen alternative VirtualBox. The main advantage of QEMU is the ability to set up a guest OS environment across host systems with vastly different processor architectures. It is not something I required however. Since I am only looking for a virtualization solution in the way of setting up deployment and testing environments, I require not flexibility but rather robust support and easy guest-host system communication. I need something that works well with contemporary systems and does it out of the box.
    \item WSL - WIndows Subsystem for Linux - The decision to disqualify was a simple one; it would not allow me to set up a MacOS guest system.
    \item Oracle VirtualBox - This virtualization solution allowed me to easily set up and test several Linux distributions as well as, with a little modification, a MacOS installation, all with a reasonable level of hardware support including 3d hardware acceleration, bi-directional clipboard sharing and performance comparable to running the guest system natively on my hardware.
\end{enumerate}
\subsection{Target Operating Systems}
\begin{enumerate}
    \item Windows - My host system is already running Windows 10 and considering the issues with  sourcing a fully licensed other version like Win 8, 7, Vista or XP (I would not consider anything older as even XP is now depreciated), I have made the rational choice of sticking to what is already running on my development system.
    \item MacOS - There are no alternatives to MacOS for deploying on Mac computers. Here, and I simply sourced an ISO of MacOS Mojave which I updated to the latest version upon installation on my Virtual Machine.
    \item Linux - I have initially considered Arch Linux as the distribution for testing my research environment on Linux because it is highly customizable. In practice this means it is also bare bones out of the box, requires a lot of impractical configuration, and may not necessarily reflect what a common contemporary Linux user is likely to be running. I have further tested OpenSUSE, Mint, Debian and Lubuntu before finally deciding on Ubuntu as the distribution to use. Ubuntu is a Debian derivative, it is largely compatible with that system as well as its own forks like Lubuntu, Kubuntu and Mint. It is also probably the most commonly used Linux distribution amongst non-technical users and therefore likely to be used by a Linux user of the research environment that I am preparing. In most other cases, it is at least very likely to be compatible. I decided to use Ubuntu as the lowest common denominator.
\end{enumerate}
\subsection{Research Environment Base - Web Browser with Hover Dictionary Support}
\begin{enumerate}
    \item Edge - There is no dictionary plugin for Edge which could meet my requirements. Support for third party stores and their plugins can be enabled on Edge allowing it to use Chrome plugins but Edge itself is tied to the Windows ecosystem and could not be deployed on Linux or a Mac without emulation.
    \item Internet Explorer - There is no dictionary plugin for Edge which could meet my requirements. IE does not even support plug-in in the contemporary sense.
    \item Chrome, Chromium and other derivatives - Google Chrome is not an open source project even though it is based on open source Chromium. Furthermore, I have privacy concerns when it comes to using Google products as they have no ideological commitment to respecting user privacy. What speaks in favour of Chrome or Chromium is that ports of my chosen hover dictionary plugins are available and that Hypothesis annotation platform also offers a plugin for Chrome where there is none for Firefox.
    \item Opera - Opera is also based on Chromium and therefore incorporates Google code base, raising the same concerns for user privacy.
    \item Safari - Safari is limited to MacOS and therefore not adequate as a base for a research environment targeting multiple platforms.
    \item Firefox - I have chosen Firefox as it is a free, open source web browser and there is a precedent for pre-packaging a pre-configured, portable, multi-platform and re-branded deployment package in Tor Project. The problem with Firefox is that Hypothesis annotation platform only offers a bookmarklet for browsers other than Chrome. This may limit the functionality, and I am now thinking of looking into preparing two alternative packages with a research environment based on Chromium for users not concerned with privacy, and based on Firefox for those who are.
\end{enumerate}
\subsection{English Japanese Hover Dictionary with De-inflection}
\begin{enumerate}
    \item Rikaichan, the first plugin of this kind is now depreciated and no longer maintained. It does not work in the current versions of Firefox after Firefox migrated to a new plugin architecture.
    \item Rikaisama was a fork of Rikaichan featuring extended functionality targeting Japanese language learners. It is also no longer maintained and not compatible with the current main branch of Firefox.
    \item Rikaichamp is a port of the original Rikaichan to the new Firefox plugin architecture. It requires minimum configuration and has a low system resource footprint making it a good choice for multi-platform deployment. Rikaichamp should perform reasonably well even on under powered systems.
    \item Yomichan has all the same practical functionality as Rikaichap but does not come with built-in dictionary files. This is not an issue since I will be pre-packaging it with Firefox and will make sure that it is fully configured, the advantage of this option is that it enables the user to purchase and install 3rd party dictionary files which may be better or more specialized (for example technical, academic or slang dictionaries) than the built-in dictionary included in Rikaichamp. Yomichan is noticeably slower and less responsive in terms of user experience, it feels noticeably sluggish even on high end systems like mine. I would like to include it as an option for power users and use Rikaichamp as the default dictionary solution.
\end{enumerate}
\subsection{Packaging and Deployment Solution}
\begin{enumerate}
    \item Mozilla Institutional Deployment - Reading the documentation I have noticed that this tool specifically revolves around creating msi packages which are fundamentally tied into the Windows ecosystem and therefore not suitable for multi-platform deployment.
    \item Portable zip archive - If I can simply pre-configure my Firefox based research environment, I can simply put it into a zip archive which can be extracted and used on virtually all contemporary operating systems. If I can have one set of configuration files and platform independent plugins accompanied by binaries for each of the target platforms, this will allow the user to keep the software on a USB stick and simply plug it in and run it on any of the supported operating systems.
    \item Tor Project - This item is not exactly a solution but rather an example of the kind of deployment solution I would like to develop in terms of having a re-branded and pre-configured Firefox based work environment. There is one major difference in that the Tor Project offers zip archives that are platform specific whereas I will be looking at having one archive that can work on as many platforms as possible.
\end{enumerate}
\subsection{Annotation Solution}
\begin{enumerate}
    \item Liner - Liner is small, robust and offers online sync but the functionality is limited for non-paying users. There are privacy concerns regarding the free version and the collaboration functionality is limited.
    \item Hypothesis directly markets itself with features specifically targeting education and research users making it a good choice. It is also free which removes the problems that I have with Liner. The only issue is that they do not offer a plugin for Firefox which is my primary choice of base software to build my research environment on.
\end{enumerate}
\end {document}
