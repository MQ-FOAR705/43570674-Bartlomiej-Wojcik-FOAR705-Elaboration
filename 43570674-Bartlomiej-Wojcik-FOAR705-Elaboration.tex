\documentclass{article}
\usepackage[utf8]{inputenc}

\title{FOAR705 Elaboration}
\author{Bart Wojcik}
\date{6 September 2019}

\begin{document}

\maketitle

\section{Introduction}
Upon consultation with Brian (I thought my initial project was trivial to accomplish), I have decided to expand the scope my project and the final goal will be to pre-configure and pre-package a ready for distribution and deployment collaborative work environment for researchers and students working with digital texts in Japanese language. The initial goal to make it easier and quicker to complete multiple dictionary look-ups will still remain a large part of the project. The expanded functionality will include annotation and collaboration functionality that would allow, for example, a research student and supervisor to annotate, markup and collaboratively communicate while working on a text from using a pre-configured digital workspace.

\section{Project Requirements}
\begin{enumerate}
    \item Self-contained digital workspace
    \item Built-in enhanced dictionary
    \item Annotation or mark-up functionality
    \item Share-abilty of annotations between collaborators
\end{enumerate}
\section{Packaging and Deployment Requirements and Options}
\begin{enumerate}
    \item Final research environment must be prepacked to reduce set-up by user to minimum
    \item Final package must offer the option for portable (USB) and local (local installation) deployments
    \item Final environment must be multiplatform (at the moment I am looking at packaging for Windows and Linux systems, in the case of Linux I am limiting myself to Debian and derivative distributions such as Ubuntu, Mint etc. I will look into the possibility of prepackaging for Mac OS.)
\end{enumerate}

\section{Technology Required}
\begin{enumerate}
    \item Virtual Machine - I need to be able to develop and test on Windows, Linux and MacOSX (or whatever they call it these days, I'll stick to calling it MacOSX for now). Oracle VirtualBox should suffice, I am already using it to run Ubuntu 19.04 on a Windows 10 host which takes care of two of my target operating systems. I just need to get my hands on a MacOSX iso which looks to be obtainable upon some initial web searches.
    \item Target Platforms - Windows, Linux, MacOS
    \item Firefox browser - Firefox is a flexible and open source web browser which offers binary executables for my target platforms. Firefox will form the basis and the core user interface of the final environment, I will need a way to remove and replace the Firefox branding and replace it with something specific to this project, for the sake of non-technical user sanity, I don't want the end user to be deploying a Firefox branded package. If the user already has Firefox installed, this could cause confusion. I will be looking into preparing a package in a way similar to the way Tor Project (https://www.torproject.org/) packages theirs. I will look at packaging this up in such a way that when a portable installation is used, it can be executed on all three of my target platforms without any loss of configuration or workspace data.
    \item Japanese - English hover dictionary plugins for Firefox - I would like the final environment to allow the user to choose between a faster and more stable Rikaichamp dictionary and the more customisable Yomichan dictionary plugin for power users.
    \item Annotation solution - I have two candidates here: Hypothesis \newline (https://web.hypothes.is/) and Liner (https://getliner.com/)
    \item Packaging and deployment solution for the final package. I believe that Mozilla Institutional Deployment \newline (https://wiki.mozilla.org/Firefox:2.0\verb|_|Institutional\verb|_|Deployment) should supply some initial ideas if not a solution.
    \item Any other collaboration tools which I may come across in the coming month that can be integrated into the environment.
\end{enumerate}

\section{Goals}
In simple terms, the goal is to give a non-technical user working with digital text in Japanese a pre-configured software package that makes it easier to read the text and collaborate on it, requires minimal technical knowledge to set-up and works on common operating systems out of the box.

\section {Technology Testing}
\subsection{Virtual Machine}
\begin{enumerate}
    \item QEMU - I have prior experience with QEMU. Almost a decade ago, I was using QEMU to emulate x86 architecture processors on a PowerPC system. I have found QEMU to be flexible but not as robustly supported as my chosen alternative VirtualBox. The main advantage of QEMU is the ability to set up a guest OS environment across host systems with vastly different processor architectures. It is not something I required however. Since I am only looking for a virtualization solution in the way of setting up deployment and testing environments, I require not flexibility but rather robust support and easy guest-host system communication. I need something that works well with contemporary systems and does it out of the box.
    \item WSL - WIndows Subsystem for Linux - The decision to disqualify was a simple one; it would not allow me to set up a MacOS guest system.
    \item Oracle VirtualBox - This virtualization solution allowed me to easily set up and test several Linux distributions as well as, with a little modification, a MacOS installation, all with a reasonable level of hardware support including 3d hardware acceleration, bi-directional clipboard sharing and performance comparable to running the guest system natively on my hardware.
\end{enumerate}
\subsection{Target Operating Systems}
\begin{enumerate}
    \item Windows - My host system is already running Windows 10 and considering the issues with  sourcing a fully licensed other version like Win 8, 7, Vista or XP (I would not consider anything older as even XP is now depreciated), I have made the rational choice of sticking to what is already running on my development system.
    \item MacOS - There are no alternatives to MacOS for deploying on Mac computers. Here, and I simply sourced an ISO of MacOS Mojave which I updated to the latest version upon installation on my Virtual Machine.
    \item Linux - I have initially considered Arch Linux as the distribution for testing my research environment on Linux because it is highly customizable. In practice this means it is also barebones out of the box, requires a lot of impractical configuration, and may not necessarily reflect what a common contemporary Linux user is likely to be running. I have further tested OpenSUSE, Mint, Debian and Lubuntu before finally deciding on Ubuntu as the distribution to use. Ubuntu is a Debian derivative, it is largely compatible with that system as well as its own forks like Lubuntu, Kubuntu and Mint. It is also probably the most commonly used Linux distribution amongs non-technical users and therefore likely to be used by a Linux user of the research environment that I am preparing. In most other cases, it is at least very likely to be compatible. I decided to use Ubuntu as the lowest common denominator.
\end{enumerate}
\subsection{Research Environment Base}
\begin{enumerate}
    \item 
\end{enumerate}

\end {document}
